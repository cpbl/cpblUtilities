% This is an example of how to display a cpbl-tables .tex file without
% including the whole cpbl-tables.sty style file.  It obviously covers
% only the simplest / a single set of possible options for the display.

\documentclass{article}
\usepackage[table]{xcolor}
\usepackage{relsize}
\begin{document}
\newcommand{\ctNtabCols}{}
\newcommand{\ctFirstHeader}{}
\newcommand{\ctSubsequentHeaders}{}
\newcommand{\ctFirstHeaderTrans}{}
\newcommand{\ctSubsequentHeadersTrans}{}
\newcommand{\ctBody}{}
\newcommand{\ctBodyTrans}{}
\newcommand{\ctCaption}{}
\newcommand{\ctCaptionTrans}{}
\newcommand{\ctStartTabular}{}
\newcommand{\ctStartTabularTrans}{}
\newcommand{\ctStartLongtable}{}
\newcommand{\ctStartLongtableTrans}{}
\newcommand{\ctNtabColsTrans}{}
\newcommand{\sltrheadername}[1]{#1}
\newcommand{\sltcheadername}[1]{#1}
\newcommand{\cpblbottomrule}{}
\newcommand{\cpbltoprule}{}
\newcommand{\showSEs}[2]{#1} % To show SEs
    \newcommand{\coefse}[1]{{\smaller\smaller (#1)}}

    \definecolor{cSignifOne}{rgb}{.92,1,.92}
    \definecolor{cSignifTwo}{rgb}{.78,1,.78}
    \definecolor{cSignifThree}{rgb}{.5,1,.5}
    \definecolor{cSignifThousandth}{rgb}{.6,1,0} % .3 1 .3
    \newcommand{\colourswrapSigTenPercent}[1]{#1\cellcolor{cSignifOne}}
    \newcommand{\colourswrapSigFivePercent}[1]{#1\cellcolor{cSignifTwo}}
    \newcommand{\colourswrapSigOnePercent}[1]{#1\cellcolor{cSignifThree}}
    \newcommand{\colourswrapSigOneThousandth}[1]{#1\cellcolor{cSignifThousandth}}
    \newcommand{\wrapSigTenPercent}{\colourswrapSigTenPercent}
    \newcommand{\wrapSigFivePercent}{\colourswrapSigFivePercent}
    \newcommand{\wrapSigOnePercent}{\colourswrapSigOnePercent}
    \newcommand{\wrapSigOneThousandth}{\colourswrapSigOneThousandth}

    \newcommand{\YesMark}{{\sc y}}%\textcolor{blue}$\checkmark$}}
    \newcommand{\cpblColourLegend}{{\footnotesize Significance:~\begin{tabular}{cccc}
        \wrapSigOneThousandth{0.1\%}~~~~&
        \wrapSigOnePercent{1\%}~~~~&
        \wrapSigFivePercent{5\%}~~~~&
        \wrapSigTenPercent{10\%}
        \\ \end{tabular} }}



    \renewcommand{\ctNtabCols}{19}
    \renewcommand{\ctFirstHeader}{\ctSubsequentHeaders \hline }
    \renewcommand{\ctSubsequentHeaders}{\cpbltoprule
	&\multicolumn{9}{c|}{\sltcheadername{degree}}	&\multicolumn{9}{c|}{\sltcheadername{fourway}}\\ 
	&\multicolumn{3}{c|}{\sltcheadername{bg}}	&\multicolumn{3}{c|}{\sltcheadername{county}}	&\multicolumn{3}{c|}{\sltcheadername{state}}	&\multicolumn{3}{c|}{\sltcheadername{bg}}	&\multicolumn{3}{c|}{\sltcheadername{county}}	&\multicolumn{3}{c|}{\sltcheadername{state}}\\ 
	&\sltcheadername{(1)}	&\sltcheadername{(2)}	&\sltcheadername{(3)}	&\sltcheadername{(4)}	&\sltcheadername{(5)}	&\sltcheadername{(6)}	&\sltcheadername{(7)}	&\sltcheadername{(8)}	&\sltcheadername{(9)}	&\sltcheadername{(10)}	&\sltcheadername{(11)}	&\sltcheadername{(12)}	&\sltcheadername{(13)}	&\sltcheadername{(14)}	&\sltcheadername{(15)}	&\sltcheadername{(16)}	&\sltcheadername{(17)}	&\sltcheadername{(18)}\\ 
}
    \renewcommand{\ctBody}{\sltrheadername{mean slope}	& \wrapSigOneThousandth{$-$.29}	& 	& \wrapSigOnePercent{$-$.025}	& \wrapSigOneThousandth{$-$.43}	& 	& \wrapSigOnePercent{.14}	& \wrapSigOneThousandth{$-$.46}	& 	& $-$.47	& \wrapSigOneThousandth{$-$.22}	& 	& .004	& \wrapSigOneThousandth{$-$.34}	& 	& \wrapSigOneThousandth{.19}	& \wrapSigOneThousandth{$-$.33}	& 	& $-$.030\\ 
\showSEs{	& \coefse{.004}	& 	& \coefse{.009}	& \coefse{.022}	& 	& \coefse{.049}	& \coefse{.083}	& 	& \coefse{.30}	& \coefse{.003}	& 	& \coefse{.008}	& \coefse{.021}	& 	& \coefse{.046}	& \coefse{.080}	& 	& \coefse{.33} \\ }{}
\sltrheadername{fraction  $>$10$^{\small\circ}$}	& 	& \wrapSigOneThousandth{$-$.30}	& \wrapSigOneThousandth{$-$.28}	& 	& \wrapSigOneThousandth{$-$.47}	& \wrapSigOneThousandth{$-$.60}	& 	& \wrapSigOneThousandth{$-$.43}	& .009	& 	& \wrapSigOneThousandth{$-$.24}	& \wrapSigOneThousandth{$-$.24}	& 	& \wrapSigOneThousandth{$-$.38}	& \wrapSigOneThousandth{$-$.56}	& 	& \wrapSigOneThousandth{$-$.34}	& $-$.31\\ 
\showSEs{	& 	& \coefse{.003}	& \coefse{.010}	& 	& \coefse{.017}	& \coefse{.050}	& 	& \coefse{.093}	& \coefse{.31}	& 	& \coefse{.003}	& \coefse{.008}	& 	& \coefse{.017}	& \coefse{.048}	& 	& \coefse{.097}	& \coefse{.36} \\ }{}
\hline 
obs.	& 72876	& 72876	& 72876	& 2383	& 2383	& 2383	& 51	& 51	& 51	& 72876	& 72876	& 72876	& 2383	& 2383	& 2383	& 51	& 51	& 51\\ 
$R^2$(adj)	& .082	& .091	& .091	& .187	& .223	& .224	& .196	& .172	& .180	& .049	& .055	& .055	& .118	& .146	& .150	& .088	& .099	& .080\\ 
}
    
    % Default caption:
    \renewcommand{\ctCaption}{ {\color{blue} first stage-subset-byscale \footnotesize \cpblColourLegend    } }
    % This .tex file is meant to be called by something from
    % cpblTables.sty. If it is not, then output something crude:
    \ifx\@ctUsingWrapper\@empty
    %Code to be executed if the macro is undefined
    \begin{table}
    \begin{tabular}{lccc|ccc|ccc|ccc|ccc|ccc|}
    \ctFirstHeader
    \ctBody
    \end{tabular}
    \end{table}
    \else
    %Code to be executed if the macro IS defined
    \fi

    % Better yet, for version "CA" of cpblTables, define methods so that the format need not be specified in the call.
    \renewcommand{\ctStartTabular}{\begin{tabular}{lccc|ccc|ccc|ccc|ccc|ccc|}}
    \renewcommand{\ctStartLongtable}{\begin{longtable}[c]{lccc|ccc|ccc|ccc|ccc|ccc|}}
    

% BEGIN TRANSPOSED VERSION

    \renewcommand{\ctNtabColsTrans}{6}
    \renewcommand{\ctFirstHeaderTrans}{\cpbltoprule 
	&	&\begin{sideways}\sltcheadername{mean slope}\end{sideways}	&\begin{sideways}\sltcheadername{fraction  $>$10$^{\small\circ}$}\end{sideways}	&\begin{sideways}\sltcheadername{obs.}\end{sideways}	&\begin{sideways}\sltcheadername{$R^2$(adj)}\end{sideways}\\ 
\hline
}
    \renewcommand{\ctSubsequentHeadersTrans}{\hline 
 	&	&\begin{sideways}\sltcheadername{mean slope}\end{sideways}	&\begin{sideways}\sltcheadername{fraction  $>$10$^{\small\circ}$}\end{sideways}	&\begin{sideways}\sltcheadername{obs.}\end{sideways}	&\begin{sideways}\sltcheadername{$R^2$(adj)}\end{sideways}\\ 
\hline
}
    \renewcommand{\ctBodyTrans}{\sltheadernum{(1)}	& \sltrheadername{bg}	& \wrapSigOneThousandth{$-$.29}	& 	& 72876	& .082\\ 
\showSEs{	& 	& \coefse{.004}	& 	& 	&  \\ }{}
\sltheadernum{(2)}	& \sltrheadername{bg}	& 	& \wrapSigOneThousandth{$-$.30}	& 72876	& .091\\ 
\showSEs{	& 	& 	& \coefse{.003}	& 	&  \\ }{}
\sltheadernum{(3)}	& \sltrheadername{bg}	& \wrapSigOnePercent{$-$.025}	& \wrapSigOneThousandth{$-$.28}	& 72876	& .091\\ 
\showSEs{	& 	& \coefse{.009}	& \coefse{.010}	& 	&  \\ }{}
\bottomrule
\sltheadernum{(4)}	& \sltrheadername{county}	& \wrapSigOneThousandth{$-$.43}	& 	& 2383	& .187\\ 
\showSEs{	& 	& \coefse{.022}	& 	& 	&  \\ }{}
\sltheadernum{(5)}	& \sltrheadername{county}	& 	& \wrapSigOneThousandth{$-$.47}	& 2383	& .223\\ 
\showSEs{	& 	& 	& \coefse{.017}	& 	&  \\ }{}
\sltheadernum{(6)}	& \sltrheadername{county}	& \wrapSigOnePercent{.14}	& \wrapSigOneThousandth{$-$.60}	& 2383	& .224\\ 
\showSEs{	& 	& \coefse{.049}	& \coefse{.050}	& 	&  \\ }{}
\bottomrule
\sltheadernum{(7)}	& \sltrheadername{state}	& \wrapSigOneThousandth{$-$.46}	& 	& 51	& .196\\ 
\showSEs{	& 	& \coefse{.083}	& 	& 	&  \\ }{}
\sltheadernum{(8)}	& \sltrheadername{state}	& 	& \wrapSigOneThousandth{$-$.43}	& 51	& .172\\ 
\showSEs{	& 	& 	& \coefse{.093}	& 	&  \\ }{}
\sltheadernum{(9)}	& \sltrheadername{state}	& $-$.47	& .009	& 51	& .180\\ 
\showSEs{	& 	& \coefse{.30}	& \coefse{.31}	& 	&  \\ }{}
\bottomrule
\sltheadernum{(10)}	& \sltrheadername{bg}	& \wrapSigOneThousandth{$-$.22}	& 	& 72876	& .049\\ 
\showSEs{	& 	& \coefse{.003}	& 	& 	&  \\ }{}
\sltheadernum{(11)}	& \sltrheadername{bg}	& 	& \wrapSigOneThousandth{$-$.24}	& 72876	& .055\\ 
\showSEs{	& 	& 	& \coefse{.003}	& 	&  \\ }{}
\sltheadernum{(12)}	& \sltrheadername{bg}	& .004	& \wrapSigOneThousandth{$-$.24}	& 72876	& .055\\ 
\showSEs{	& 	& \coefse{.008}	& \coefse{.008}	& 	&  \\ }{}
\bottomrule
\sltheadernum{(13)}	& \sltrheadername{county}	& \wrapSigOneThousandth{$-$.34}	& 	& 2383	& .118\\ 
\showSEs{	& 	& \coefse{.021}	& 	& 	&  \\ }{}
\sltheadernum{(14)}	& \sltrheadername{county}	& 	& \wrapSigOneThousandth{$-$.38}	& 2383	& .146\\ 
\showSEs{	& 	& 	& \coefse{.017}	& 	&  \\ }{}
\sltheadernum{(15)}	& \sltrheadername{county}	& \wrapSigOneThousandth{.19}	& \wrapSigOneThousandth{$-$.56}	& 2383	& .150\\ 
\showSEs{	& 	& \coefse{.046}	& \coefse{.048}	& 	&  \\ }{}
\bottomrule
\sltheadernum{(16)}	& \sltrheadername{state}	& \wrapSigOneThousandth{$-$.33}	& 	& 51	& .088\\ 
\showSEs{	& 	& \coefse{.080}	& 	& 	&  \\ }{}
\sltheadernum{(17)}	& \sltrheadername{state}	& 	& \wrapSigOneThousandth{$-$.34}	& 51	& .099\\ 
\showSEs{	& 	& 	& \coefse{.097}	& 	&  \\ }{}
\sltheadernum{(18)}	& \sltrheadername{state}	& $-$.030	& $-$.31	& 51	& .080\\ 
\showSEs{	& 	& \coefse{.33}	& \coefse{.36}	& 	&  \\ }{}
\bottomrule
}
    
    % Default caption:
    \renewcommand{\ctCaptionTrans}{ {\color{blue} first stage-subset-byscale \footnotesize \cpblColourLegend    } }

    % Better yet, for version "CA" of cpblTables, define methods so that the format need not be specified in the call.
    \renewcommand{\ctStartTabularTrans}{\begin{tabular}{lp{3cm}*{4}{r}}}
    \renewcommand{\ctStartLongtableTrans}{\begin{longtable}[c]{lp{3cm}*{4}{r}}}
    

%
% Following does Csimple style, not transposed.
\begin{table}
\setlength\tabcolsep{1pt}
%\footnotesize
\hspace{-2cm}\begin{tabular}{|c|ccc|ccc|ccc|ccc|ccc|ccc|ccc|} \hline \ctFirstHeader \ctBody \hline \end{tabular} \cpblColourLegend
\end{table}




\afterpage{\clearpage
\renewcommand{\longTableFooter}{} %%{\multicolumn{#4}{l}{#}}
\ifcpblTableUseTransposed\cpblTablesSwitchTransposed\fi
\begin{longtable}[c]{|c|ccc|ccc|ccc|ccc|ccc|ccc|ccc|}

\ctFirstHeader

\endfirsthead
\ctSubsequentHeaders
\hline \endhead 
 \hline \multicolumn{\ctNtabCols}{r}{\longtableContinueFooter}  \endfoot  \longTableFooter \endlastfoot 
\ctBody
\cpblbottomrule 
%%%  \renewcommand{\ntabcols}{#5}
%%\usetinytablefont 
\end{longtable}
\ctResetDefinitionsClosing
}% afterpage



\end{document}



%\newcommand{
